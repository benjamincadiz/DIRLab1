\documentclass{article}

\usepackage{fancyhdr} % Required for custom headers
\usepackage{lastpage} % Required to determine the last page for the footer
\usepackage{extramarks} % Required for headers and footers
\usepackage[usenames,dvipsnames]{color} % Required for custom colors
\usepackage{graphicx} % Required to insert images
\usepackage{listings} % Required for insertion of code
\usepackage{courier} % Required for the courier font
\usepackage{url}

%----------------------------------------------------------------------------------------
%	SPANISH LANGUAGE
%----------------------------------------------------------------------------------------
\usepackage{lmodern}
\usepackage[T1]{fontenc}
\usepackage[spanish,activeacute]{babel}
\usepackage[utf8]{inputenc} % Para poder incluir tildes y ñ automáticamente

%----------------------------------------------------------------------------------------
%	TODOs AND NOTES
%----------------------------------------------------------------------------------------

\usepackage{xargs}                      % Use more than one optional parameter in a new commands
\usepackage[pdftex,dvipsnames]{xcolor}  % Coloured text etc.
\usepackage[colorinlistoftodos,prependcaption,textsize=tiny]{todonotes}
\newcommandx{\unsure}[2][1=]{\todo[linecolor=red,backgroundcolor=red!25,bordercolor=red,#1]{#2}}
\newcommandx{\change}[2][1=]{\todo[linecolor=blue,backgroundcolor=blue!25,bordercolor=blue,#1]{#2}}
\newcommandx{\info}[2][1=]{\todo[linecolor=OliveGreen,backgroundcolor=OliveGreen!25,bordercolor=OliveGreen,#1]{#2}}
\newcommandx{\improvement}[2][1=]{\todo[linecolor=Plum,backgroundcolor=Plum!25,bordercolor=Plum,#1]{#2}}
\newcommandx{\thiswillnotshow}[2][1=]{\todo[disable,#1]{#2}}

%----------------------------------------------------------------------------------------
%	MARGIN, HEADER AND FOOTER
%----------------------------------------------------------------------------------------
% Margins
\topmargin=-0.45in
\evensidemargin=0in
\oddsidemargin=0in
\textwidth=6.5in
\textheight=9.0in
\headsep=0.25in

\linespread{1.1} % Line spacing

% Set up the header and footer
\pagestyle{fancy}
\lhead{\textbf{Memoria Análisis de Redes.}} % Top left header
\rhead{\textbf{Benjamín Cádiz de Gracia.}} % Top right header
\lfoot{\lastxmark} % Bottom left footer
\rfoot{Página\ \thepage\ de\ \protect\pageref{LastPage}} % Bottom right footer
\renewcommand\headrulewidth{0.4pt} % Size of the header rule
\renewcommand\footrulewidth{0.4pt} % Size of the footer rule

\setlength\parindent{0pt} % Removes all indentation from paragraphs

%----------------------------------------------------------------------------------------
%	CODE INCLUSION CONFIGURATION
%----------------------------------------------------------------------------------------
\definecolor{gray97}{gray}{.97}
\definecolor{gray75}{gray}{.75}
\definecolor{gray45}{gray}{.45}
\definecolor{comments}{RGB}{0,100,0}

\lstset{ frame=Ltb,
     framerule=0pt,
     aboveskip=0.5cm,
     framextopmargin=3pt,
     framexbottommargin=3pt,
     framexleftmargin=0.4cm,
     framesep=0pt,
     rulesep=.4pt,
     backgroundcolor=\color{gray97},
     %
     stringstyle=\ttfamily\color{comments},
     showstringspaces = false,
     basicstyle=\footnotesize\ttfamily,
     commentstyle=\itshape\color{gray45},
	 keywordstyle=\bfseries\color{blue},
     %
     numbers=left,
     numbersep=15pt,
     numberstyle=\tiny,
     numberfirstline = false,
     breaklines=true,
   }
   
\lstdefinestyle{ConfigFiles}{% define own style
  language={[LaTeX]TeX},
  basicstyle=\small\ttfamily,
  linewidth=0.9\linewidth,
  breaklines=true,
  keywordstyle=\color{blue}\bfseries,
  identifierstyle=\color{orange},
  commentstyle=\color{cyan},
  backgroundcolor=\color{OliveGreen},
  tabsize=2,
  morekeywords = {parameter},
}

\lstdefinestyle{consola}
   {basicstyle=\scriptsize\bf\ttfamily,
    backgroundcolor=\color{gray75},
   }
 
\lstdefinestyle{C}
   {language=C,
   }


\begin{document}

\begin{titlepage}
\begin{center}
\vspace*{-1in}
\begin{figure}[htb]
\begin{center}
\includegraphics[width=5cm]{./images/uclm_logo.eps} 
\hspace*{1.5in}
\includegraphics[width=5cm]{./images/esi_logo.eps}
\end{center}
\end{figure}
\end{center}
\begin{center}
UCLM - Escuela Superior de Informática - Ciudad Real\\
\vspace*{0.6in}
\vspace*{0.2in}
\begin{Large}
\textbf{Diseño e Infraestructura de Redes.} \\
\textbf{Memoria Práctica 1.} \\
\end{Large}
\vspace*{0.3in}
\vspace*{0.3in}
\rule{80mm}{0.1mm}\\
\vspace*{0.1in}
\begin{large}
Hecho por: \\
Benjamín Cádiz de Gracia. \\
\vspace*{0.3in}
Fecha: Febrero de 2018.\\
\end{large}
\end{center}

\end{titlepage}
\tableofcontents
%Cuerpo del documento
\newpage

\begin{center}
\section{ Red Toroide.}
\end{center}
\subsection{Enunciado.}
Dado un archivo con nombre \textit{datos.dat}, cuyo contenido es una lista de valores separados por comas, nuestro programa realizará lo siguiente:

El proceso de rank 0 destribuirá a cada uno de los nodos de un toroide de lado L,los L x L numeros reales que estarán contenidos en el archivo datos.dat.
En caso de que no se hayan lanzado suficientes elementos de proceso para los datos del programa, éste emitirá un error y todos los procesos finalizarán.
En caso de que todos los procesos han recibido su correspondiente elemento, comenzará el proceso normal del programa.
Se pide calcular el elemento menor de toda la red, el elemento de proceso con rank 0 mostrará en su salida estándar el valor obtenido.

La complejidad del algoritmo no superará O(raiz\_cuadrada(n)) Con n número de elementos de la red.

\subsection{Planteamiento de la solución.}

Para obtener nuestra solución vamos a utilizar una \textit{red toroide}. La \textit{red toroide} es una malla que permite comunicar un dato con un vecino suyo. Nuestro planteamiento sera utilizar este tipo de red para mediante, primero por filas y despues por columnas, obtener el mínimo de todos los valores asociados a cada nodo de la red.


\subsection{Diseño del programa}

El programa está dividido en: 
\begin{list}{•}{}
\item \textbf{Main}. Un main principal, donde su función será la del algoritmo principal de obtener el número mas pequeño.
\item \textbf{Obtener vecinos}. Una función donde podremos calcular la posición de todos y cada uno de los vecinos. Simplemente será una función para que cada nodo tenga una referencia de donde están todos sus vecinos.
\item \textbf{Distribuir}. Un método cuya función será leer el fichero donde contiene los datos, y por cada dato se distribuirá con el fin de asignarlo a su nodo correspondiente.
\item \textbf{Obtener}. Este método se complementa con el anterior, ya que recibirá el dato enviado y además, se asignará el dato.
\end{list}

\subsection{Flujo de datos en la red.}

\subsection{Código fuente.}
\begin{lstlisting}
//
//  red_toroide.c
//  
//
//  Created by Benjamin Cadiz de Gracia on 27/2/18.
//

#include <mpi.h>
#include <stdio.h>
#include <stdlib.h>

#define DATOS "Datos.dat"
#define L 3

void obtain_neighbors(int* north, int* south, int* east, int* west, int rank,
                      int numtasks){
    
    *south = (rank + L) % numtasks;
    
    *north = (rank - L) % numtasks;
    if(*north < 0){
        *north += numtasks;
    }
    
    *east = rank + 1;
    if(*east % L == 0){
        *east -= L;
    }
    
    *west = (rank - 1) % numtasks;
    if(*west % L == L-1){
        *west += L;
    }else if(*west == -1){
        *west = L-1;
    }
}

void distribuir(int numtask, int rank, float* token){
    int i = 0;
    int rc;
    FILE* file;
    float num[16];
    MPI_Status status;
    MPI_Request request;
    float element=0;
    
    //Leemos el archivo.
    if ((file=fopen(DATOS,"r"))==NULL){
        fprintf(stderr, "Error opening the file\n");
        exit(EXIT_FAILURE);
    }else
        while(!feof (file) & (i < numtask)){
            fscanf (file,"%g,", &element);
            
            //Enviamos en el primer elemento.
            if(i == rank){
                *token = element;
            }else{
                rc = MPI_Isend(&element, 1, MPI_FLOAT, i, 1, MPI_COMM_WORLD, &request);
                if (rc != MPI_SUCCESS) {
                    printf("Send error in task %d\n", rank);
                    MPI_Finalize();
                    exit(1);
                }
                MPI_Wait(&request, &status);
            }
            i++;
        }
        fclose(file);
}
void obtain_token(float* token, int rank){
    int rc;
    
    MPI_Status status;
    rc = MPI_Recv(token, 1, MPI_FLOAT, 0, MPI_ANY_TAG, MPI_COMM_WORLD, &status);
    if (rc != MPI_SUCCESS) {
        printf("Receive error in task %d\n", rank);
        MPI_Finalize();
        exit(1);
    }
}

int main(int argc, char** argv) {
    float token;
    FILE* file;
    
    int rank,size,numtask;
    int  north,south,east,west;
    
    // Inicializacion de MPI.
    MPI_Init(&argc, &argv);
    MPI_Comm_rank(MPI_COMM_WORLD, &rank);
    MPI_Comm_size(MPI_COMM_WORLD, &numtask);
    
    //Obtenemos vecinos.
    obtain_neighbors(&north,&south,&east,&west,rank,numtask);
    
    //Distribuimos los elementos del fichero en nuestros nodos.
    if(rank == 0){
        distribuir(numtask,rank, &token);
    }else{
        obtain_token(&token, rank);
    }
    
    //Realizamos el algoritmo.
    int rc;
    float new_token;
    MPI_Status status;
    for( int i =0;i<L-1;i++){
        rc = MPI_Send(&token,1, MPI_FLOAT, north,i,MPI_COMM_WORLD);
        if (rc != MPI_SUCCESS){
            printf("Receive error in task %d\n", rank);
            MPI_Finalize();
            exit(1);
        }
        rc = MPI_Recv(&new_token,1,MPI_FLOAT,south,MPI_ANY_TAG,MPI_COMM_WORLD,&status);
        if (rc != MPI_SUCCESS){
            printf("Receive error in task %d\n", rank);
            MPI_Finalize();
            exit(1);
        }
        if(new_token < token){
            token = new_token;
        }
    }
    
    for( int i =0;i<L-1;i++){
        rc = MPI_Send(&token,1, MPI_FLOAT, west,i,MPI_COMM_WORLD);
        if (rc != MPI_SUCCESS){
            printf("Receive error in task %d\n", rank);
           	MPI_Finalize();
            exit(1);
        }
        rc = MPI_Recv(&new_token,1,MPI_FLOAT,east,MPI_ANY_TAG,MPI_COMM_WORLD,&status);
        if (rc != MPI_SUCCESS){
            printf("Receive error in task %d\n", rank);
            MPI_Finalize();
            exit(1);
        }
        if(new_token < token)
            token = new_token;
    }
    
    if(rank ==0)
        printf("The minimun value is:  %g \n",token);
    
    MPI_Finalize();
    return EXIT_SUCCESS;
}


\end{lstlisting}

\subsection{Compilación y ejecución.}

Debemos situarnos en la carpeta donde están los archivos \textit{red\_toroide.c}, \textit{red\_hipercubo.c} y \textit{\textbf{Makefile}}. Allí abriremos una terminal y ejecutaremos:
\begin{list}{•}{}
\item Primero limpiamos todos los posibles ejecutables que haya en el proyecto.
\begin{lstlisting}
make clean
\end{lstlisting}
\item En segundo lugar compilaremos bien \textit{red\_toroide.c} o bien todo.
\begin{lstlisting}
make red_toroide		//Compila únicamente el toroide.
make all					//Compila todo
\end{lstlisting}
\item En tercer lugar lanzaremos el ejecutable.
\begin{lstlisting}
make run_toroide
\end{lstlisting}
Cabe decir que por defecto nos viene 9 hilos para la ejecución.
\end{list}
 
\begin{center}
\section{Red Hipercubo.}
\end{center}
\subsection{Enunciado.}
Dado un archivo con nombre datos.dat, cuyo contenido es una lista de valores separados por comas, nuestro programa realizará lo siguiente:

El proceso de rank 0 distribuirá a cada uno de los nodos de un Hipercubo de dimensión D, los $2^D$ números reales que estarán contenidos en el archivo datos.dat. En caso de que no se hayan lanzado suficientes elementos de proceso para los datos del programa, éste emitirá un error y todos los procesos finalizarán.
En caso de que todos los procesos han recibido su correspondiente elemento, comenzará el proceso normal del programa.
Se pide calcular el elemento mayor de toda la red, el elemento de proceso con rank 0 mostrará en su salida estándar el valor obtenido. 

La complejidad del algoritmo no superará O(logaritmo\_base\_2(n)) )) Con n número de elementos de la red.

\subsection{Planteamiento de la solución.}

Para obtener nuestra solución vamos a utilizar una \textit{red toroide}. La \textit{red toroide} es una malla que permite comunicar un dato con un vecino suyo. Nuestro planteamiento sera utilizar este tipo de red para mediante, primero por filas y despues por columnas, obtener el mínimo de todos los valores asociados a cada nodo de la red.


\subsection{Diseño del programa}

El programa está dividido en: 
\begin{list}{•}{}
\item \textbf{Main}. Un main principal, donde su función será la del algoritmo principal de obtener el número mas pequeño.
\item \textbf{Obtener vecinos}. Una función donde podremos calcular la posición de todos y cada uno de los vecinos. Simplemente será una función para que cada nodo tenga una referencia de donde están todos sus vecinos.
\item \textbf{Distribuir}. Un método cuya función será leer el fichero donde contiene los datos, y por cada dato se distribuirá con el fin de asignarlo a su nodo correspondiente.
\item \textbf{Obtener}. Este método se complementa con el anterior, ya que recibirá el dato enviado y además, se asignará el dato.
\end{list}

\subsection{Flujo de datos en la red.}

\subsection{Código fuente.}

\begin{lstlisting}
//
//  red_hipercubo.c
//  
//
//  Created by Benjamin Cadiz de Gracia on 16/3/18.
//

#include <mpi.h>
#include <stdio.h>
#include <stdlib.h>
#include <math.h>

#define DATOS "Datos.dat"
#define D 4

int obtain_neighbor(int rank, int dimension){
    int neighbor, node;
    for(node = 0; node < (int)pow(2,D); node++){
        if((rank ^ node) == (int)pow(2,dimension - 1)){
            neighbor = node;
            break;
        }
    }
    return neighbor;
}
void distribuir(int numtask, int rank, float* token){
    int i = 0;
    int rc;
    FILE* file;
    float num[16];
    MPI_Status status;
    MPI_Request request;
    float element=0;
    
    //Leemos el archivo.
    if ((file=fopen(DATOS,"r"))==NULL){
		fprintf(stderr, "Error opening the file\n");
        exit(EXIT_FAILURE);
    }else
        while(!feof (file) & (i < numtask)){
            fscanf (file,"%g,", &element);
            
            //Enviamos en el primer elemento.
            if(i == rank){
                *token = element;
            }else{
                rc = MPI_Isend(&element, 1, MPI_FLOAT, i, 1, MPI_COMM_WORLD, &request);
                if (rc != MPI_SUCCESS) {
                    printf("Send error in task %d\n", rank);
                    MPI_Finalize();
                    exit(1);
                }
                MPI_Wait(&request, &status);
            }
            i++;
        }
    fclose(file);
}
void obtain_token(float* token, int rank){
    int rc;
    
    MPI_Status status;
    rc = MPI_Recv(token, 1, MPI_FLOAT, 0, MPI_ANY_TAG, MPI_COMM_WORLD, &status);
    if (rc != MPI_SUCCESS) {
        printf("Receive error in task %d\n", rank);
        MPI_Finalize();
        exit(1);
    }
}

int main(int argc, char** argv) {
    float token;
    
    int rank,size,numtask;
    FILE* file;
    
    // Inicializacion de MPI.
    MPI_Init(&argc, &argv);
    MPI_Comm_rank(MPI_COMM_WORLD, &rank);
    MPI_Comm_size(MPI_COMM_WORLD, &numtask);
    
    
    //Distribuimos los elementos del fichero en nuestros nodos.
    if(rank == 0){
        distribuir(numtask,rank, &token);
    }else{
        obtain_token(&token, rank);
    }
    
    //Realizamos el algoritmo.
    int rc;
    float new_token;
    MPI_Status status;
    for( int i =1;i<=D;i++){
        rc = MPI_Send(&token,1, MPI_FLOAT,obtain_neighbor(rank, i),i,MPI_COMM_WORLD);
        
        if (rc != MPI_SUCCESS){
            printf("Receive error in task %d\n", rank);
            MPI_Finalize();
            exit(1);
        }
        rc = MPI_Recv(&new_token,1,MPI_FLOAT,obtain_neighbor(rank, i),MPI_ANY_TAG,MPI_COMM_WORLD,&status);
        if (rc != MPI_SUCCESS){
            printf("Receive error in task %d\n", rank);
            MPI_Finalize();
            exit(1);
        }
        if(new_token > token){
            token = new_token;
        }
    }
    
    if(rank ==0)
        printf("The maximum value is:  %g \n",token);
    MPI_Finalize();
    return EXIT_SUCCESS;
}


\end{lstlisting}
  
\subsection{Compilación y ejecución.}

Debemos situarnos en la carpeta donde están los archivos \textit{red\_toroide.c}, \textit{red\_hipercubo.c} y \textit{\textbf{Makefile}}. Allí abriremos una terminal y ejecutaremos:
\begin{list}{•}{}
\item Primero limpiamos todos los posibles ejecutables que haya en el proyecto.
\begin{lstlisting}
make clean
\end{lstlisting}
\item En segundo lugar compilaremos bien \textit{red\_hipercubo.c} o bien todo.
\begin{lstlisting}
make red_hipercubo	//Compila únicamente el toroide.
make all					//Compila todo
\end{lstlisting}
\item En tercer lugar lanzaremos el ejecutable.
\begin{lstlisting}
make run_toroide
\end{lstlisting}
Cabe decir que por defecto nos viene 16 hilos para la ejecución.
\end{list}

\begin{center}
\section{Makefile.}
\end{center}

\begin{lstlisting}
#!/usr/bin/make -f
# -*- mode:makefile -*-

CC := mpicc
DIREXEC := ejecutables/
RUN := mpirun

dirs:
	mkdir -p $(DIREXEC)

red_toroide: dirs
	$(CC) red_toroide.c -o $(DIREXEC)toroide

red_hipercubo: dirs
	$(CC) red_hipercubo.c -o $(DIREXEC)hipercubo

all: red_toroide red_hipercubo

run_hipercubo: red_hipercubo
	$(RUN) -n 16 $(DIREXEC)hipercubo

run_toroide: red_toroide
	$(RUN) -n 9 $(DIREXEC)toroide

clean:
	rm -rf $(DIREXEC)



\end{lstlisting}
 
 

\newpage
\end{document}